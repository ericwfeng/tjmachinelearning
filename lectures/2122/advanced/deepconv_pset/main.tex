\documentclass{article}
\usepackage[utf8]{inputenc}
\usepackage{hyperref}
\hypersetup{
    colorlinks=true,
    linkcolor=blue,
    filecolor=magenta,      
    urlcolor=cyan,
    pdftitle={Overleaf Example},
    pdfpagemode=FullScreen,
    }

\title{ResNet Problem Set \\ Total Points: 10}
\author{Tarushii Goel}
\date{2020}

\usepackage[letterpaper, margin=1in]{geometry}
\usepackage{natbib}
\usepackage{graphicx}
\usepackage{amsmath}



\begin{document}

\begin{center}
\fbox{\fbox{\parbox{5.5in}{\centering
Problem Set: Inception \& Resnet \\
Due date:  \\
Total Points: 10\\
If you run out of room for an answer, use scratch paper and staple it to this sheet.}}}
\end{center}
 
\vspace{5mm}
 
\makebox[\textwidth]{Name and Grade:\enspace\hrulefill}
 
\vspace{5mm}

\begin{enumerate}
	\item (8 points) Calculate the output of the following 2-layer residual block: 
	\begin{equation}
	x = \begin{bmatrix}
		1 \\
		7 \\
		4 \\
		\end{bmatrix},
	w_1 = \begin{bmatrix}
		2 & 5 & 2\\
		10 & 8 & 9\\
		2 & 6 & 1\\
		\end{bmatrix},
	b_1 = \begin{bmatrix}
		6 \\
		9 \\
		4 \\
		\end{bmatrix},
	w_2 = \begin{bmatrix}
		1 & 1 & 1\\
		10 & 10 & 10\\
		3 & 3 & 3\\
		\end{bmatrix},
	b_2= \begin{bmatrix}
		7 \\
		3 \\
		2 \\
		\end{bmatrix},
	\sigma(x) = \tanh(x)
	\end{equation}
	For work, you must at least show all operations you applied on $x$ to get your answer and the final matrix. However, you do not need to show intermediate matrices and may use a calculator to calculate them. 
	(For clarity, the weights are structured as: 
		$$ \begin{bmatrix}
		w_{11} & w_{21} & w_{31}\\
		w_{12} & w_{22} & w_{32}\\
		w_{13} & w_{23} & w_{33}\\
		\end{bmatrix} $$
		where for $w_{xy}$, $x$ is the number of the input node and $y$ is the number of the output node)
	\pagebreak 
	\item (2 points) Why do we use ResNets? 

\end {enumerate}
\end{document}
